\documentclass[12pt]{article}
\usepackage{geometry}

\newcommand{\g}[1]{{\color{blue}{#1}}}
\newcommand{\plr}[1]{{\color{green}{#1}}}

\geometry{a4paper} 

\title{Annotated spatial popgen bibliography}

%%%%%%%%%%%%
\begin{document}

\maketitle
\newpage

\begin{enumerate}
%%%
\item Bradburd, Ralph, Coop (2013) - BEDASSLE
\subitem detecting barriers against spatial backdrop

%%%
\item Bradburd, Ralph, Coop (2016) - SpaceMix
\subitem detecting admixture against spatial backdrop
\subitem defines admixture as something transgressive against geographic expectations,
 rather than transgressive against discrete ``population history".

%%%
\item Bradburd, Coop, Ralph (2018) - conStruct
\subitem defining ``clusters" against continuous spatial backdrop
\subitem implicit null that IBD explains everything
 
%%%
\item Wilkins (2004) - A Separation-of-Timescales Approach to the Coalescent in a Continuous Population
\subitem spatial exploration of separation of timescales (with range margins)

%%%
\item Barton \& Wilson (1995) - Genealogies and geography
\subitem introduces diffusion approximation to model the distribution of coalescent times 
in a homogeneous 2D environment

%%%
\item Barton, Depaulis, \& Etheridge (2002) - Neutral Evolution in Spatially Continuous Populations
\subitem introduces a general recursion for the probability of identity in state of two individuals 
sampled from a population subject to mutation, migration, and random drift in a two-dimensional continuum

%%%
\item Barton, Kelleher, \& Etheridge (2013) - A New Model for Extinction and Recolonization in Two Dimensions: Quantifying Phylogeography
\subitem introduces conceptual model of population dynamics that side-steps 
the spatial discrepancy in the distribution of individuals between coalescent models 
and forward-time models (highlighted in Felsenstein 1975)

%%%
\item Barton, Etheridge, Kelleher, V\'{e}ber (2013) - Inference in two dimensions: Allele frequencies versus lengths of shared sequence blocks
\subitem introduces 2 approaches (one based on allele frequency covariance, 
the other on length of shared sequence blocks) 
to inference of neighbourhood size and dispersal rate.

%%%
\item Felsenstein (1975) - A Pain in the Torus: Some Difficulties with Models of Isolation by Distance
\subitem demonstrates that Mal\'{e}cot's model of IBD leads to clumping of individuals, 
and that, unmodified, it can therefore not be used to describe/simulate datasets with biological realism.

%%%
\item Petkova, Novembre, Stephens (2015) - Visualizing spatial population structure with estimated effective migration surfaces
\subitem identification of local barriers to average gene flow
\subitem tesselates the landscape, places genotyped individuals on the landscape, 
estimates what is essentially ``resistance" across edges to match observed divergence

%%%
\item Al-Asadi, Petkova, Stephens, Novembre (2018) - Estimating recent migration and population size surfaces
\subitem EEMS, but estimated using matrices of genetic similarity 
derived from sharing of segments of different lengths, 
which are informative about different slices of time in the past 
(like localPCA, but plugged into EEMS).

%%%
\item Pritchard, Stephens, \& Donnelly - STRUCTURE
\subitem OG model of discrete population structure 
\subitem see also Alexander, Novembre, and Lange (2009, 2011), Lawson et al (2012)

%%%
\item Reich et al (2009) - Reconstructing Indian population history
\subitem introduced f-stats (see also Patterson et al 2012 and Peter 2016)

%%%
\item Nagylaki (1998) - The Expected Number of Heterozygous Sites in a Subdivided Population
\subitem nice presentation of structured coalescent

%%%
\item Novembre \& Slatkin (2009) - Likelihood-based inference in isolation-by-distance models 
using the spatial distribution of low-frequency alleles
\subitem present statistical method for inferring product of population density and dispersal (aka $\rho\sigma^2$)

%%%
\item PCA papers
\subitem Patterson et al (2006), Novembre et al (2008), Novembre \& Stephens (2008), McVean (2009)

%%%
\item coalHMM papers - reviewed in Pence et al (2018)
\subitem can be used for estimating $N_e$

%%%
\item Peter \& Slatkin (2013) - Detecting range expansions from genetic data
\subitem 

%%%
\item Peter \& Slatkin (2015) - The effective founder effect in a spatially expanding population
\subitem 

%%%
\item Ralph \& Coop (2013) - The Geography of Recent Genetic Ancestry across Europe
\subitem 

%%%
\item Lundgren \& Ralph (2018) - Are populations like a circuit? 
The relationship between isolation by distance and isolation by resistance
\subitem 

%%%
\item Aguillon et al (2017) - Deconstructing isolation-by-distance: the genomic consequences of limited dispersal
\subitem 

%%%
\item Ringbauer, Coop, \& Barton (2017) - Inferring recent demography from isolation by distance of long shared sequence blocks
\subitem 

%%%
\item Ringbauer et al (2018) - Estimating Barriers to Gene Flow from Distorted Isolation-by-Distance Patterns
\subitem 

%%%
\item Whitlock \& McCauley (1999) - 
\subitem 

%%%
\item Wang \& Bradburd (2014) - Isolation by Environment
\subitem 

%%%
\item Wright (1943) - Isolation by distance
\subitem 

%%%
\item Wright (1946) - Isolation by distance under diverse systems of mating
\subitem 

%%%
\item Cayuela et al (2018) - Demographic and genetic approaches to study dispersal in wild animal populations: A methodological review
\subitem 

%%%
\item Charlesworth (2009) - Effective population size and patterns of molecular evolution and variation
\subitem 

%%%
\item Haller, B.C., \& Messer, P.W. (2017). SLiM 2
\subitem 

%%%
\item Flagel, Brandvain, \& Schrider (2018). The Unreasonable Effectiveness of Convolutional Neural
Networks in Population Genetic Inference
\subitem 

%%%
\item Lawson, van Dorp, \& Falush (2018) - A tutorial on how not to over-interpret STRUCTURE and ADMIXTURE bar plots
\subitem 

%%%
\item McRae 2006, 2007 - Isolation by resistance
\subitem 

%%%
\item Li \& Ralph 2018 - Local PCA shows how the effect of population structure differs along the genome
\subitem 

%%%
\item Mal\'{e}cot (1948, tr 1969)
\subitem

\end{enumerate}

\end{document}