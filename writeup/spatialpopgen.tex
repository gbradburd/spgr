% template.tex, dated April 5 2013
% This is a template file for Annual Reviews 1 column Journals
%
% Compilation using ar-1col.cls' - version 1.0, Aptara Inc.
% (c) 2013 AR
%
% Steps to compile: latex latex latex
%
% For tracking purposes => this is v1.0 - Apr. 2013

\documentclass{ar-1col}

\setcounter{secnumdepth}{4}

% Metadata Information
\jname{Xxxx. Xxx. Xxx. Xxx.}
\jvol{AA}
\jyear{YYYY}
\doi{10.1146/((please add article doi))}


% Document starts
\begin{document}

% Page header
\markboth{Bradburd}{Spatial Population Genetics}

% Title
\title{Spatial Population Genetics: It's About Time}


%Authors, affiliations address.
\author{Gideon S. Bradburd,$^1$ Peter L. Ralph, $^2$
\affil{$^1$Ecology, Evolutionary Biology, and Behavior Group, Department of Integrative Biology, Michigan State University, East Lansing, MI, USA, 48824;
email: bradburd@msu.edu}
\affil{$^2$Institute of Ecology and Evolution, Departments of Mathematics and Biology, University of Oregon, Eugene, OR 97403}
}

%Abstract
\begin{abstract}
Abstract text, approximately 150 words. 
\end{abstract}

%Keywords, etc.
\begin{keywords}
population genetics, isolation by distance, population structure, local adaptation
%keywords, separated by comma, no full stop, lowercase
\end{keywords}
\maketitle

%Table of Contents
\tableofcontents


% Heading 1
\section{INTRODUCTION}

\begin{itemize}
\item The field of population genetics is shaped by a continuing conversation 
between theory, methods, and data.
\item With some notable exceptions (wright, malecot), 
much of early theory was population-centric, 
both informing and informed by early empirical datasets.
	\subitem Again, with some notable exceptions, e.g. Dobzhansky 47.
\item In reality, well-delineated demes or clearly defined discrete populations are rare.
Rather, organisms are living, moving, reproducing, and dying somewhere in space, 
all connected to each other by a pedigree that spans the history of all species.
\item \emph{Something happened}, 
and now there's a growing body of theory and datasets 
that approaches this reality.
\item Methods are playing catch-up, 
and it's a very exciting time to be a spatial population geneticist.
\item What is spatial popgen?
\item What is this review?
\subitem This review is organized around a small number of 
fundamental questions about the biology and history of organisms 
that underlie many of the things we want to know about our study species.
\end{itemize}

\section{The spatial pedigree}
\begin{itemize}
\item What is the spatial pedigree and why is it useful?
\item Spatial pedigree figure
\end{itemize}

\section{Where are they?}
\begin{itemize}
\item When considering a species, 
perhaps the first thing we want to know is where they are: 
where they can be found, and how many of them there are there.
If we are only interested in the modern distribution of a species, 
we can answer this question by going outside and counting individuals.
However, we may be interested in the distribution of individuals over time, 
especially as the distribution of abundance of many species is changing rapidly 
in response to anthropogenic disturbance.
\item The history of abundance is recorded in the spatial pedigree.
\subitem What is this on the spatial pedigree
\item We can use data that were generated under this spatial pedigree 
to get a peek at the process
\subitem Types of data, methods (SMC-ish methods, Ne methods)
\end{itemize}

\section{How do they move?}

\section{How have things changed over time?}


\section{Are there groups of them?}


\section{Exciting directions for the field}

%Disclosure
\section*{DISCLOSURE STATEMENT}
If the authors have nothing to disclose, the following statement will be used: The authors are not aware of any affiliations, memberships, funding, or financial holdings that
might be perceived as affecting the objectivity of this review. 

% Acknowledgements
\section*{ACKNOWLEDGMENTS}
Brandvain, Marge, JoNo, Graham, Schemske, friendly reviewers

\end{document}

%%%%%%%%%%%%%%%%%%%
%%%%%%%%%%%%%%%%%%%
% TEXT GRAVEYARD
%%%%%%%%%%%%%%%%%%%
%%%%%%%%%%%%%%%%%%%


Population genetics stands out as a discipline 
in which the development of theory was far ahead of empirical possibilities.
Lewontin (1974) described the early state of the field as 
``an immensely rich and powerful theory with virtually no suitable facts on which to operate."
For example, the foundations of modern population genetic theory were laid down in XXX, 
some XXX years before the first empirical, 
molecular measurements of genetic variation were made.
Since then, advances in genotyping and sequencing 
have allowed empirical datasets to begin to catch up to theory, 
and increasingly, to spur the development of methods to study those new data.
The study of the spatial partitioning of population genetic variation 
is an excellent example of this progression.
A large body of early work, 
especially that contributed by Wright and Mal\'ecot, 
developed theory for populations in an explicit spatial context.
%MORE DETAIL ON THEORY HERE?
%\begin{itemize}
%	\item early theory (continuous and discrete)
%	\begin{itemize}
%		\item wright, malecot, maruyama, nagylaki, felsenstein, slatkin, barton, etheridge, etc.
%         		\item something about isolation by distances and the First Law of Geography (Tobler 1970):
%		            ``everything is related to everything else, but near things are more related than distant things."
%	\end{itemize}
%	\item early empirical popgen out of spatial context (limited by sampling)
%	\begin{itemize}
%		\item e.g., Lewontin and Hubby (``find em and grind em")
%		\item Maybe some stuff on early phenotypable genotypes (pepper moths? )
%		\item chromosomes types (dobzhansky and others)
%		\item But see also wright's work on Linanthes and maybe early stuff on human blood group (cavalli-sforza and others?)
%		\item disconnect between early theory that anticipated the data we'd have now and data then
%\end{itemize}
However, early genotyping methods were limiting. 
The earliest empirical measurements of molecular genetic variation 
-- the inaugural ``find 'em and grind 'em" work of Lewontin and Hubby (1966) -- 
focused on data from 43 fly strains sampled from 5 locations, 
which were largely treated as experimental replicates 
(rather than studied in their geographic context).
Within a decade, genotyping by electrophoresis was being used to quantify  
genetic variation in large geographic samples (summarized in Lewontin 1974).
Each subsequent development of a new technique for measuring genetic variation, 
starting with counting visible mutations and ending in whole genome re-sequencing,
has followed a similar trajectory from few populations to many.
The major challenge of method development has been 
to build models that can keep pace with these new data 
to describe the ever greater biological realism and complexity.
Tracing the trajectory of growing richness of empirical datasets, 
early methods focused on results in a single panmictic population (ref method?), 
then iteratively relaxed the assumption of panmixia with, e.g., 
implementations of the island model (ref method?) 
and the stepping-stone model (ref method?).
In reality, well-delineated demes or clearly defined discrete populations are rare.
Rather, organisms are living, moving, reproducing, and dying somewhere in space, 
all connected to each other by a pedigree that spans the history of all species.

The current acceleration of empirical sequence generation
has spurred a revolution in the richness of datasets, 
which now requires an accompanying revolution in the development of 
methods to analyze this wealth of data. 
Large and geographically dense samples are now 
in genotype-able range in most non-model systems, 
and indeed, there are now a handful of populations in which 
\emph{every individual} has been sequenced over the past XX generations.
In addition, 
the growing ability to genotype historical or ancient individuals 
is transforming the study of population genetics, 
much of which is predicated on the idea that we can glimpse processes 
acting in the past from genetic data obtained from samples in the present, 
and for which these direct observations into the past are therefore particularly powerful.  
The focus of these datasets has, in many cases, 
shifted from discrete populations to the continuous sampling of individuals, 
especially within an explicit spatial context.
Taken together, these advances in genotyping make it 
an exciting time for method development in population genetics 
and, especially, in spatial population genetics.

The field of spatial population genetics is the study of population genetics, in space.
That is, the principal goals of population genetics -- 
to study patterns of genetic variation within and between populations and 
learn about the processes generating those patterns -- 
are the same as those of spatial population genetics. 
However, spatial population genetics as a field is particularly concerned 
with the spatial context of these patterns,
leveraging information in the position of samples 
to learn about processes governing the distribution of genetic diversity across landscapes.
Spatial population genetics allows researchers
to unite the quantitative descriptions of population genetics 
with fundamental questions about the ecology and evolution of organisms.

This review is organized around a definition of the field of spatial population genetics 
and its principal applications.
My goal is to provide an introduction for empirical researchers 
to the types of questions that can be answered 
with spatial population genetics, 
and the signals in data that can be used to answer them.
This is a rapidly changing field, 
and to a reader of this review five years from now, 
many methods currently in use may no longer be relevant 
(although hopefully my own contributions will be immortal).
So, rather than discuss the existing methods for inference in detail, 
I have instead organized this review around the principal questions in the field 
and will focus on the different types of information in genetic and genomic data 
that can be brought to bear on those questions.

\section{THE SPATIAL PEDIGREE}
Before diving in, it is useful to define a helpful concept: the spatial pedigree.
The fundamental quantity in the field of population genetics is the ancestral recombination graph (ARG), 
which describes the coalescent history of each locus in the genome (argweaver).
This set of gene genealogies is embedded within the pedigree of a species, 
in the sense that a pair of sampled alleles cannot coalesce until 
the pedigrees of the individuals they are sampled from overlap in a shared ancestor.
The pedigree of a sample is in turn embedded within some spatial context. 
Say we start with a sample of two alleles, one from each of two individuals.
If we trace their ancestors across space and back through time, 
starting from that focal pair of individuals in the present, 
we find that their pedigrees cannot overlap until their ancestors coexist in space.
(The geographic radius within which ancestors can coexist 
and pedigrees can overlap is a function of the biology of the specific organism; 
broadcast spawners or highly mobile animals can mate over much larger distances than, say, snails.)
Therefore, a useful quantity to consider in the study of 
spatial population genetics is the spatial pedigree: 
a graph denoting relatedness between all individuals in a sample, 
indexed with the geographic position of each individual.
If we had complete knowledge of the spatial pedigree of a species, 
including the parentage and location of individuals 
who contribute no ancestry to the present-day population, 
we would know everything we wished to know of the history of dispersal and gene flow, 
population structure and admixture, and selection and fitness in a system.
For example, if we wanted to know the average natal dispersal distance in the species, 
we could simply take the mean of the distribution of distances between parents and offspring, 
or if we wanted to know the dispersal rate between a pair of locations, 
we could count all the times an individual from one dispersed to the other 
over the number of generations sampled.
In practice, we never have such a complete understanding of the spatial pedigree, 
but it leaves its imprint on different types of genetic and genomic data, 
and, by collecting and analyzing these data, 
we can try to catch glimpses of it.
SPATIAL PEDIGREE FIGURE

\section{Population Structure and Admixture}
\subsection{What do researchers want to do?}
One of the first steps in the analysis of population genetic data 
is to look for and visualize population structure in the sample ? 
systematic differences in genetic similarity between groups of individuals.
The goal 
This type of analysis can take many forms, such as
dimensionality reduction plots (e.g., PCA, spacemix)
population assignment methods (structure, admixture, conStruct),
population phylogenies (treemix)
plotting pairwise divergence against pairwise geographic distance, 
chromosome painting (globetrotter, fineStructure) or ancestry tract inference (Ralph and Coop, Kelly, palamara), 
or modeling the site frequency spectrum (dadi, hickerson).
These methods analyze different types of data and employ different models of population history, 
but all seek to summarize patterns of genetic variation produced by the spatial pedigree 
connecting the individuals in the sample.
Below, I describe different types of population structure, 
explain how they arise on the spatial pedigree, 
and classify existing methods based on the data they model.

\subsection{What mechanisms underly these patterns on the spatial pedigree?}
Population structure is often well described by a pattern of isolation by distance (Wright), 
which is best summed up by the First Law of Geography (Tobler 1970):
``Everything is related to everything else, but near things are more related than distant things."
this pattern emerges from the spatial pedigree when
there is a geographically limited radius within which individuals mate, 
and spatial autocorrelation in the geographic positions of parents and offspring. 
The result is that individuals farther away from each other in the present 
tend not to overlap in their pedigrees until farther back in the past, 
leading to higher relatedness between nearby individuals 
(wright, barton-etheridge, felsenstein torus, kelleher).
Individuals distributed across a featureless plain in migration-drift equilibrium (Slatkin review) 
are expected to exhibit this continuous pattern of isolation by distance. 
How quickly this relatedness decays with distance is determined 
by the specifics of the biology of an organism: 
organisms that move themselves or their gametes over long distances, 
and that have high population densities may 
exhibit high average relatedness over long spatial lags (Wright 1946).

Sometimes, features of the landscape, 
the specific population history, 
or quirks of biology
can generate discrete population genetic structure 
above and beyond this continuous spatial process.
For example, patchily distributed demes might experience very limited migration (islands, isolated valleys, etc.), 
or range expansions might have recently brought two populations that historically 
were isolated into close proximity,
and populations in this scenario of secondary contact may 
evince some behavioral or intrinsic reproductive isolation.
Although still governed by the same spatial processes, 
these scenarios lead to spatial pedigrees in which 
the median parent-offspring distance is small relative to the total range.
Note that the mean parent-offspring distance may be much larger 
than the median if migrants rarely disperse a long distance, 
e.g. between isolated demes (Endler 1977, Rousset 2004).

Admixture is most commonly considered 
within the context of discrete population structure, 
and is used to describe populations that are composed of ancestry from 
multiple discrete groups. 
However, admixture can also occur against a backdrop of 
continuous geographic structure, 
in which case it can be interpreted as relatedness 
between the samples from a set of locations 
that is unusually high given the geographic distance that separates them (spacemix).
In both continuous and discrete contexts then, 
admixture represents a transgressive force, 
breaking across population boundaries between discrete demes 
or spatial patterns of divergence between distant locations.

In the spatial pedigree, 
we describe an individual as ``admixed" when they have recent ancestors 
from either a different discrete population or a distant location, 
and a population as admixed when many individuals in it share this property.
As has been noted elsewhere (falush, globetrotter, reich), 
the ability to identify a population as admixed 
is a function of the recency of admixture and 
the inclusion of populations descended from the sources of admixture in the modern day sample. 
Indeed, the concept of admixture becomes slipperier when considered over deeper timescales, 
as inferred sources of admixture are themselves almost certainly admixed, 
and have either been isolated for long enough that the admixture is no longer detectable, 
or, more likely, the descendants of those sources of admixture are lost in the modern sample.



%FIGURE 2: multipanel figure like Schraiber and Akey Fig 2:
%a) pedigree w/ 3 individuals highlighted (one distant, two nearby)
%b) 

\subsection{How do existing methods get at this mechanism?}



\begin{itemize}
\item Estimate Gene Flow and Dispersal
	\item What do researchers want to do?
	\subsubitem 
\item The geography of genetic variation
	\item What do researchers want to do?
\end{itemize}



























































\section{POPULATION STRUCTURE AND ADMIXTURE}
One of the principal aims of spatial population genetics is to describe 
and visualize the distribution of genetic variation in a geographic sample. 
Most commonly, this means quantifying \emph{population structure} -- 
systematic differences in genetic similarity between groups of individuals.
Often, population structure is well described by a pattern of isolation by distance (Wright), 
which is perhaps best summed up by the First Law of Geography (Tobler 1970):
``Everything is related to everything else, but near things are more related than distant things."
This pattern can be considered under two paradigms: 
continuous and discrete.
In the continuous worldview, 
population structure is described continuously on a landscape, 
by, e.g., considering the rate at which relatedness between individuals decays 
with the geographic distance between them.
In the discrete worldview, 
population structure is described by defining discrete populations,
within which all individuals are more or less equally related, 
and between which there are varying degrees of divergence.
Defined populations may then be used for 
management and conservation (``management units" Moritz),
systematics and taxonomy (e.g., describing species or subspecies, ref?)
or delineating evolutionary units of analysis for further study (e.g., GWAS).
These paradigms are often invoked to explain the same 
pattern of isolation by distance (IBD), 
and, indeed, are not mutually exclusive.
Under both the continuous and discrete worldviews,
individuals or populations can be \emph{admixed}: 
composed of a combination of ancestry 
from two or more discrete groups or locations.

\subsection{Approaches}
There are a variety of existing approaches for inferring 
and visualizing population structure and admixture, 
including model-based clustering approaches 
(e.g., STRUCTURE, ADMIXTURE, fineStructure, faststructure, tess, geneland, conStruct),
estimating population phylogenies (e.g, TREEMIX,mixmapper?),
reduced-dimensionality analyses (PCA, novembre and stephens, spacemix, SPAmix),
and several heuristics or statistics that range from the simple (Fst, f-statistics) 
to the complex (roloff, globetrotter).

These methods employ different models or inference algorithms 
and leverage different types of signal 
in genetic and genomic data to learn about 
the spatial history of population structure and admixture.
To see where that signal comes from, 
it is useful to first consider how these processes manifest in the spatial pedigree.

\subsection{Continuous and Discrete Structure, and Admixture}

\paragraph{Continuous Structure}
Continuous, geographically distributed population structure is the simplest to consider: 
this pattern emerges from the spatial pedigree when
there is a geographically limited radius within which individuals mate, 
and spatial autocorrelation in the geographic positions of parents and offspring. 
The result is that individuals farther away from each other in the present 
tend not to overlap in their pedigrees until farther back in the past, 
leading to higher relatedness between nearby individuals 
(wright, barton-etheridge, felsenstein torus, kelleher).

\paragraph{Discrete Structure}
Discrete structure between populations arises 
when the coalescent time of alleles sampled 
from individuals within a particular group are, on average, lower than 
that of a pair of alleles sampled from individuals in different groups (maruyama, nagylaki?). 
Because the coalescent is embedded within the pedigree, 
this can arise when individuals within a cluster 
have more recent pedigree overlap with each other than 
they do with individuals between clusters.
Indeed, clusters are defined by this property, 
without which ``clusters'' would only be meaningless labels 
 randomly assigned to individuals.
In a geographic context, discrete structure is most frequently (and parsimoniously) 
ascribed to a reduction in migration between groups 
due to geographic separation between patchily distributed demes.
	-something about how this can be due to modern or historical isolation.
		-Note also that equilibrium vs. nonequlibrium 
			-for example, discrete population structure can be generated due to historical \emph{or} modern isolation, 

\paragraph{Admixture}
Admixture is most commonly considered 
within the context of discrete population structure, 
and is used to describe populations that are composed of ancestry from 
multiple discrete groups. 
However, admixture can also occur against a backdrop of 
continuous geographic structure, 
in which case it can be interpreted as relatedness 
between the samples from a set of locations 
that is unusually high given the geographic distance that separates them (spacemix).
In both continuous and discrete contexts then, 
admixture represents a transgressive force, 
breaking across population boundaries between discrete demes 
or spatial patterns of divergence between distant locations.

In the spatial pedigree, 
we describe an individual as ``admixed" when they have recent ancestors 
from either a different discrete population or a distant location, 
and a population as admixed when many individuals in it share this property.
As has been noted elsewhere (falush, globetrotter, reich), 
the ability to identify a population as admixed 
is a function of the recency of admixture and 
the inclusion of populations descended from the sources of admixture in the modern day sample. 
Indeed, the concept of admixture becomes slipperier when considered over deeper timescales, 
as inferred sources of admixture are themselves almost certainly admixed, 
and have either been isolated for long enough that the admixture is no longer detectable, 
or, more likely, the descendants of those sources of admixture are lost in the modern sample.

\subsection{The Signals in Data}
The pedigree connecting a set of samples under consideration, 
along with the geographic position of each ancestor and descendant, 
is almost never known; 
to identify the signatures left by continuous and discrete population structure and admixture, 
we must rely on inferences from patterns of genetic and genomic variation 
shaped by the coalescent history embedded within that spatial history. 
Methods for detecting population structure and admixture 
generally model patterns in one of two types of data: 
single-locus and two-locus. 

\subsubsection{Single-locus}
In a single-locus analysis, 
each locus is treated as an independent draw from the coalescent, 
and the data considered are the frequencies of alleles at one or more loci.
These data are generally considered in one of two ways:
patterns of allele frequency divergence or the site frequency spectrum.

\paragraph{Site frequency spectrum}
The site frequency spectrum (SFS) is a summary of population genomic data 
that describes the number of derived alleles at a particular frequency 
in a sample of individuals.
Under neutrality, and within a panmictic population of constant size, 
the proportion of derived alleles found in $i$ out of $n$ genotyped chromosomes 
is proportional to $1/i$ (Wakeley).
This expected distribution is derived by considering a mutational process 
that drops mutations at random onto the coalescent history of the samples.
In a population of constant size under no selection, 
the majority of the length of the coalescent tree is in terminal branches. 
The majority of mutations dropped onto this tree at random will therefore be unshared
(only found on a single chromosome). 
Relatively few will fall on deeper branches, 
which would make them shared by more individuals in the sample.
To develop an intuition for how deviations from panmixia can leave an imprint on the SFS, 
it is useful to return to the spatial pedigree and the coalescent it contains.

As described above, discrete population structure in the spatial pedigree manifests as, 
on average, more recent pedigree overlap between individuals in the same group 
compared to pairs of individuals from different groups.
This leads to a coalescent tree with short intra-group branch lengths 
and long inter-group branch lengths, 
as individuals in different groups have to go farther back in time until their 
ancestors co-occur in the same group and their pedigrees can overlap (nagylaki, maruyama?).
As more of the total tree branch length is in deep branches between groups, 
a greater number of mutations on the tree are found in all members of a group, 
but not shared between groups, 
leading to an unusual number of intermediate-frequency alleles 
in the SFS.

SFS in continuous space?
	-no work that I know of
		-presumably shape of SFS is strongly affected by the sampling configuration relative to relevant biological quantities
		-e.g., clumped spatial sampling might be indistinguishable from discrete pop structure
	-a parameterized spatial model should generate an expected coalescent history, 
		so I guess you could calculate the likelihood of some observed SFS under a particular spatial model?

Admixture and the SFS?
	-Admixture describes a transgression in the spatial pedigree: 
		an ancestor of provenance that is unusual given time and space separating it from its descendant.

-what type of glimpse does the SFS offer into the spatial pedigree?
-Methods that use the SFS

\paragraph{Allele frequency divergence}

Drift and selection act locally to generate differentiation between 
discrete populations or locations in continuous space, 
and migration acts to homogenize those differences that arise.
Under drift, 
the frequency of a given allele at a particular location 
is not expected to change through time, 
but the variance around that expectation grows with the amount of time 
the individuals at that location are isolated, 
and the expected difference in the frequency of that allele between 
two samples grows (non-linearly) with the amount of isolation between them.
Within populations or locations that are less isolated from each other, 
allele frequencies are less free to corkscrew off on independent evolutionary trajectories, 
as any changes that arise locally are quickly equalized by migration between them.

Drift is random, 
and may be quite slow to differentiate samples if effective population sizes are large,
so the amount of divergence in frequency at any particular locus 
may not be informative about their isolation. 
However, measured in aggregate across many independent loci, 
divergence in allele frequencies can detect even very subtle population structure (Patterson 2006).
Allele frequency divergence offers a glimpse at gene flow averaged 
over the history of the spatial pedigree.


-Methods that use allele frequency divergence
 
\paragraph{Two-locus}

In contrast to a single locus analysis, 
a two-locus analysis leverages information contained 
in the spatial distribution of loci on the genome.


-two locus Methods


\subsection{Spatial Assignment}
	-SCAT (and others?)
	-identifying origins of expansions/invasions
	-estimating wright's neighborhood size

\section{ESTIMATING DISPERSAL AND GENE FLOW}

	\subsection{Identifying barriers and quantifying resistance}
	present: barriers and dispersal (box: gene flow vs. dispersal)
		-IBD-squared, globetrotter, actual pedigrees, MAPS, harald's stuff
			-field of ``landscape genetics"
			-historical data
			-noneq vs. eq
	past: historical migration (and asymmetries)
	-EEMS, BEDASSLE, MAPS, harald's stuff
	\subsection{Inferring demography}
	-SFS models, PSMC/SMSC (covariance in histories)
		-two pop problems vs. many inds
	-colonization/expansion routes
	-identifying invasion sources
	-abundant center hypothesis
\section{THE GENETIC BASIS OF LOCAL ADAPTATION}
	-could also be ``the geography of genetic variation"
		-focus shifts from patterns of divergence in allele frequencies
		 to how the frequency of a particular allele is geographically distributred 
		 (novembre and dirienzo)
		 	-e.g., skin color alleles, duffy/malaria
	-BAYenv and similar
	-gene dropping down pedigrees, if known
\section{THE FUTURE}
	past - continuity (maybe in 1st section?)
	attempting to infer pedigree (palamara, rasmus, kelly)
	sims (slim and ABC, slim and machine learning)
	cheaper gps tags
	infinite data
	



%\subsection{Gene flow, dispersal, and migration}
%	inds who leave no ancestry may still matter for epidemiological/ecological/conservation
%	Whitlock and McCauley
%	Sokal and Oden 1978
%\subsection{genealogy and geography}
%	-how does pedigree overlap relate to relatedness?

Brief history of work:
\begin{itemize}
	\item early theory (continuous and discrete)
	\begin{itemize}
		\item wright, malecot, maruyama, nagylaki, felsenstein, barton, etheridge
         		\item something about isolation by distances and the First Law of Geography (Tobler 1970):
		            ``everything is related to everything else, but near things are more related than distant things."
	\end{itemize}
	\item early empirical popgen out of spatial context (limited by sampling)
	\begin{itemize}
		\item e.g., Lewontin and Hubby (``find em and grind em")
		\item Maybe some stuff on early phenotypable genotypes (pepper moths?)
		\item But see also wright's work on Linanthes and maybe early stuff on human blood group (cavalli-sforza and others?)
		\item disconnect between early theory that anticipated the data we'd have now and data then
	\end{itemize}
	\item Now lots of theory and methods and datasets, 
	\begin{itemize}
		\item but advent of historical/ancient DNA (gamechanger for popgen generally)
			\item  discovery that lots of populations are not continuous through time
		\item drives home a few points:
		\begin{itemize}
			\item we are probably rarely in a migration-drift equilibrium world
			\item genetic variation may be disassociated with geographic context
			\item important to recognize the possibility of heterogeneity of process associated with location through time
		\end{itemize}
	\end{itemize}
		\item Given the availability of these datasets and these advances, I thought it timely to review the blah blah blah
\end{itemize}

FIGURES:
	spatial pedigree
		space, time, pedigree, coalescence, IBD tract, drift
	modern vs. historical isolation (zipping up/zipping down)
	dispersal surfaces AND continuity changing through time

NOTES FROM GRAHAM:
	theory v inference
	one locus v two locus
	one pop to two pops
	marry msmc to harald's stuff

	look at Novembre and DiRienzo paper (2009)
	cavalli-sforza (expansion stuff from pca, or marriage records in Italy, cited in IBD2, harald, or cavalli-sforza remembrances)
	fisher (blood group variants sign of viking invasion)
	mathieson and mcvean pepper moth through time?

	flow from new tech to a small number of pops to a large number of pops (latter of which requires more theory/inference)
	collecting/scattering

NOTES FROM DOUG:
	theory so far ahead of empirical possibilities
	no genotyping, just mendelian traits
	TRYING TO ESTABLISH A SPATIAL NULL
	linanthes perrii (wright's work derived in response to epling's data)
		-wright was only interested in shifting balance
			-expectation of differentiation in space given effective population sizes
			-maybe patterns in linanthes are just explained by drift and differences in pop sizes
			-electrophoresis lead to a rediscovery of Fst, now that they could calculate it
				-annual review by marilyn loveless and jim hamrick - review data for Fst
					-what plants have high fst?
	collected paper on ecology and evolutionary biology of invasions (grosberg has a paper on it)
		-no evidence for bottlenecking associated invasion
EXTENT TO WHICH GEOGRAPHIC STRUCTURE IN SELECTION IS STRONG ENOUGH TO OVERCOME DRIFT
continuous vs. discontinuous approach (are organisms every truly continuously distributed?)
speciation: geography of speciation
	-linking spatial distribution of adaptive variation to speciation processes
		-how much of the baggage of history is maintained
	-geographic range and range limits
		-range margins (surfing) swamping (kirkpatrick and barton)? running out of variation?
-on methods: what kinds of markers do we need to answer this question?
-using population structure to identify regions under selection
	-nordborg bergelson
	-using clines in allele frequency abundance to identify mechanism of selection?
	-send doug an outline

NOTES FROM BVAIN:
That body of theory has been remarkably useful;
however nearly all of population genetics theory starts with panmictic populations.
It has been tweaked, to accommodate inbreeding and demic structure 
(for example, the island model or the stepping-stone model)
This theory has been close and gotten us far
but a major challenge is alligning theory with reality
no populations 
no k
just orgs living and dieing and moving in space
and then you can work through ealy ibd
and discrete dems
up to where we are 
- build on history to say why we need this
and its exciting
and is to do w data

NOTES FROM PETER:
where are they (make me a map of popn density and mean fecundity)
	-draw a circle on a map and how many inds are int
		if that just's about today, just go count
	-how many have been there over some time, 
		-can't just count, but it's recorded in the pedigree
	-
how do they move
what do they do once they're there
are there groups of them?
	-clustering is a crutch that humans use to think
	-could describe all of these (just historical, just ongoing, actual barriers, gnarly source/sink dynamics)
	
	
how has this changed over time
	Barton and Slatkin (hybrid zones, etc.) - nice theory that we can start to get a look at w/ good data
	few notable exceptions (e.g., dobzhansky 1947)
	
	start w/ a barrier, what's happening on a barrier??
	biases in direction of crossing barrier -> expansion/asymmetrical migration and source/sink dynamics, invasion
	
	don't start w/ pop structure, bc what the hell is pop structure
	start w/ something observable like dispersal/migration
	
	mean time to coalescence between two points in space
	
	no one actually cares about the pedigree
		-people want to know how often individuals from A go to B
			-this is answered by the pedigree, but no one actually starts out by asking about the pedigree
	
	start from biology and nonconfusing biological questions
		migration when it's not confusing (islands)
		if we knew where everyone was and what they did
		in practice we can only 
		
	barriers and biases in direction
	
	migration, kid-having, population/density
	
	lots of allele A at location 1, could be bc there were originally, or lots of inds go there and don't die, or have lots of kids there
		fitness genetic/geographic correlation
			-context-dependent kid-having
		
short thing on selection
		
		
		
% BITS OF TEXT
Frequently, admixture is described using an \emph{admixture proportion}: 
the proportion of genome in a sample that comes from a discrete group, 
which is also the probability that an allele taken at random from a sample 
come from that group.

I briefly discuss the methods used for answering these questions, 
define them in terms of processes shaping the spatial pedigree, 
and describe how different types of genetic and genomic data can offer 
(sometimes qualitatively) different insights into those processes.

Two revolutions in genotyping technology are leading to 
empirical datasets that are more representative of this reality, 
which in turn are driving the development of methods to study them.

The first revolution is one of scale.  
Setting aside humans and model systems, 
for which the wealth of available data and 
the rate of generation are simply staggering,
falling sequencing costs are bringing large and geographically dense sampling 
into genotype-able range in most non-model systems.
Indeed, there are now a handful of populations in which 
\emph{every individual} has been sequenced over the past XX generations, 
and the number of populations or species that can make a similar claim 
is sure to rise in the near future.
The second revolution is of scope.
The growing ability to genotype historical or ancient individuals 
is transforming the study of population genetics, 
much of which is predicated on the idea that we can glimpse processes 
acting in the past from genetic data obtained from samples in the present.  
